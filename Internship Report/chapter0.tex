\chapter{مقدمه}
{
صوت یکی از مهم‌ترین حالات انرژی در جهان ما می‌باشد و راه ارتباطی اصلی بسیاری از انسان‌ها و دیگر موجودات،‌ از طریق سیگنال های صوتی می‌باشد. به همین دلیل، درک و پردازش این نوع از داده‌ها، اهمیت بسیاری در عصر حاضر برای ما دارا می‌باشد. 

مدل‌های تشخیص خودکار گفتار  دسته‌ای از سیستم‌های هوش مصنوعی هستند که برای تبدیل زبان گفتاری به متن نوشتاری طراحی شده‌اند. این مدل‌ها در طیف گسترده‌ای از کاربردها، از جمله خدمات رونویسی، دستیارهای صوتی و دستگاه‌های کنترل‌شده با صدا، نقش مهمی دارند. مدل‌های  از تکنیک‌های یادگیری عمیق، مانند شبکه‌های عصبی ترانسفورمر برای پردازش داده‌های صوتی و تولید رونویسی دقیق از کلمات و عبارات گفتاری استفاده می‌کنند.

تشخیص خودکار گفتار سر به سر \LTRfootnote{End to End ASR}
یک روش در تشخیص خودکار گفتار است که در آن از یک مدل شبکه عصبی واحد برای تبدیل مستقیم زبان گفتاری به متن نوشتاری بدون نیاز به اجزای میانی مانند واج یا واحدهای زبانی استفاده می‌شود. هدف مدل‌های بازشناسی گفتار سر به سر ساده‌سازی مسیر\LTRfootnote{Pipeline} بازشناسی گفتار است و به دلیل توانایی‌شان در یادگیری نگاشت‌های پیچیده گفتار به متن محبوبیت پیدا کرده‌اند. آنها اغلب مبتنی بر معماری‌های یادگیری عمیق، مانند شبکه‌های عصبی مکرر \LTRfootnote{RNN: Recurrent Neural Network} یا ترانسفورماتورها هستند، و در دستیابی به نتایج رقابتی در تسک های تشخیص گفتار، نویدبخش نشان داده‌اند، و آنها را برای برنامه‌هایی مانند خدمات رونویسی، دستیارهای صوتی و غیره ارزشمند می‌سازند.

در شرکت عصر گویش پرداز کار مشابه‌ای در زمینه بازشناسی گفتار قبلا انجام شده است. در حال حاظر نرم افزار نویسا شرکت عصر گویش پرداز برای کاربرد های تجاری بازشناسی گفتار استفاده می‌شود. اما محدودیت نرم افزار فعلی نویسا این است که این نرم افزار با مدل های ان گرام\LTRfootnote{Ngram} آموزش داده شده است و روش های
تعبیه‌سازی کلمات\LTRfootnote{Word Embedding}
برای آموش این مدل استفاده شده است؛ مدل فعلی در درک اسامی خاص و کلمات پیچیده ضعیف عمل می‌کند زیرا در دیکشینری آن کلمه خاص ممکن است موجود نباشد. اما مدل های 
خودنگرش\LTRfootnote{Self Attention}
به روش های تعبیه سازی نیاز ندارند و خودشان می‌توانند ارزش کلمات را در جملات درک کنند. پروژه من آموزش یک مدل 
سر به سر بازشناسی گفتار فارسی
می‌باشد که دارای معماری 
خودنگرش
‌باشد.

در مطالعاتی که در این دوره کارآموزی انجام گرفت چند مقاله به منظور یافتن بهترین معماری برای تشخیص گفتار فارسی بررسی شد؛ یکی از بهترین کاندیدا ها مدل \texttt{Whisper} شرکت \texttt{AI Open} می‌باشد. این مدل توانایی بازشناسی گفتار به زبان های مختلف دارد، اما دقت خروجی آن برای زبان فارسی کم است. شرکت عصرگویش پرداز قبلا برای فاین تیون\LTRfootnote{Fine Tune} کردن این مدل اقدام کرده است اما با توجه به ماهیت 
نیمه نظارتی \LTRfootnote{Semi-Supervised}
این مدل خروجی مدل فاین توین شده مناسب نبود. با همفکری که در شرکت صورت گرفت تصمیم بر این شد که از مدل های \texttt{Conformer} یا \texttt{Branchformer} برای آموزش مدل بازشناسی گفتار فارسی استفاده شود.

در ادامه، در فصل دوم به معرفي شركت عصرگويش پرداز پرداخته و بخشي از مهم ترين محصولات و زمينه هاي فعاليت اين شركت بررسي خواهند شد همچنین به معرفی این پروژه کارآموزی خواهیم پرداخت. در فصل سوم و چهارم، تجربيات كسب شده در اين دوره كارآموزي سه ماهه، بيان خواهد شد و برخي از چالش ها و راه حل هايي كه در اين دوره ارائه شدند، بررسي خواهند شد. در فصل سوم به مباحث تئوری پروژه از جمله مدل انتخابی و دادگان اشاره خواهد شد و در فصل چهارم به تجربیات عملی و چالش های آموزش و پیاده سازی اشاره مدل اشاره خواهد شد. در نهايت در فصل پنجم، نتيجه گيري مربوط به اين دوره كارآموزي بيان خواهد شد و پيشنهاد هايي در جهت بهبود خروجی مدل ارائه شده، ذكر خواهد شد.
}