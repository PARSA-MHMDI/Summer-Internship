\chapter{معرفی محل کارآموزی و پروژه کارآموزی}

در این قسمت، به طور مختصر، شرکت عصرگویش‌پرداز معرفی شده و در ادامه محصولات اصلی شرکت و همچنین زمینه‌های فعالیت این شرکت ذکر خواهند شد.

\section{معرفی شرکت}

عصر گویش پرداز (سهامی خاص) فعال‌ترین شرکت در زمینه هوش مصنوعی و پردازش سیگنال گفتار بوده كه فعالیت خود را از ابتدای سال ۱۳۸۲ شروع كرده است. عمده محصولات و خدمات ارائه شده توسط این شرکت برای نخستین بار در کشور و به صورت حرفه‌ای در زمینه‌های پردازش و تشخیص گفتار بوده است. این شرکت با پشتوانه فنی گروهی از متخصصان کشور از دانشگاه صنعتی شریف تأسیس شد که سابقه و تجربه پژوهشی آنها در زمینه‌های مرتبط با پردازش سیگنال به چندین سال قبل از شروع رسمی فعالیت شرکت برمی‌گردد.

\subsection{محصولات شرکت}

عصرگویش پرداز پیشرو در ارائه سیستم های مبتنی بر گفتار برای زبان فارسی، محصولات مختلفی را توسعه داده است که بیشتر آنها برای نخستین بار برای زبان فارسی انجام شده و منحصراً توسط این شرکت تولید می‌شوند. برخی از محصولات این شرکت عبارتند از:
\begin{itemize}
	\item نویسا: نخستین سامانه تایپ گفتاری فارسی
	\item نیوشا: نخستین سامانه تلفن گویای هوشمند مبتنی بر گفتار
	\item آریانا: سامانه متن به گفتار فارسی با صدای طبیعی
	\item شناسا: تعیین هویت گوینده
	\item رمزآوا: احراز هویت گوینده
	\item بینا: تصویر خوان هوشمند
	\item رومند: چت بات هوشمند
	\item جویا: سامانه جستجوی عبارات و کلمات در گفتار
	\item پوشا: سامانه پنهان سازی اطلاعات در تصویر (استگانوگرافی)
	\item پدیدا: سامانه کشف تصاویر نهان نگاری شده
	\item پارسیا: اولین نرم‌افزار متـرجم گفتار به گفتار فارسی به انگلیسی/ عربی
	\item نویسیار: اولین نرم‌افزار تایپ هوشمند فارسی
	\item کارا: نخستین سامانه تشخیص فرمان صوتی برای ویندوز
\end{itemize}

\subsection{زمینه‌های فعالیت}

این شرکت امروزه دارای گروهی متخصص و منسجم از افرادی با تخصص و تجربه بالا بوده و سابقه طولانی و موفق در زمینه تحقیق و توسعه و کاربردی کردن توانمندی های پژوهشی دارد و علاوه بر ارائه محصولات مختلف در زمینه‌های هوش مصنوعی، پردازش گفتار فارسی و انگلیسی و پردازش تصویر، قادر به انجام پروژه های مختلف و ارائه خدمات در زمینه‌های مختلف نرم‌افزاری می‌باشد. از جمله زمینه‌های فعالیت این شرکت:
\begin{itemize}
	\item تولید نرم افزارها و سخت افزارهای هوشمند
	\item هوش مصنوعی و شناسایی الگو
	\item پردازش سیگنال (گفتار و تصویر)
	\item تشخیص گفتار و تایپ گفتاری (تبدیل گفتار به متن)
	\item سنتز گفتار و متن خوان (تبدیل متن به گفتار)
	\item شناسایی افراد از روی صدا
	\item پردازش زبان طبیعی
	\item بهبود كیفیت گفتار
	\item طراحی دادگان‌های گفتاری و متنی
	\item طراحی، توسعه و پشتیبانی نرم افزارهای کاربردی مرتبط
	\item سیستم‌های تلفن گویا (با قابلیت تشخیص گفتار)
	\item سامانه‌های تلفنی مبتنی بر ویپ (استریسک، الستیکس و ...)
	\item برنامه نویسی روی ریزکامپیوترها (\lr{DSP}، تلفن همراه و ...)
\end{itemize}

با توجه به نوآوری های انجام گرفته در شركت عصرگویش پرداز، این شرکت علاوه بر انتشار مقاله‌های مختلف در نشریات و کنفرانس‌های علمی ملی و بین‌المللی، دارای افتخارات و تأییدیه‌های متعددی می‌باشد.

\section{پروژه کارآموزی}
شرکت عصر گویش پرداز هرساله در فصل تابستان تعداد محدودی کارآموز از دانشجویان بهترین دانشگاه های ایران جذب می‌کند. دانشجویانی که بعد از ارسال رزومه و قبولی در مصاحبه انتخاب می‌شوند در دوره سه ماه کارآموزی شرکت عصر گویش پرداز مشغول می‌شوند. هر کارآموز باید یک یا دو پروژه از پروژه های فعال شرکت را تکمیل کند و پس از پیاده سازی و ارائه خروجی پروژه به مسئول کارآموزی گواهی اتمام کارآموزی را دریافت می‌کند.
پروژه های کارآموزی از پروژه های فعال و حل نشده شرکت می‌باشد. دانشجویان بعد از اینکه به گروه های چند نفری تقسیم شدند بر اساس دانش و علاقه آنها یک پروژه به آنها داده می‌شود و یکی از دانشجویان ارشد هوش مصنوعی دکتر صامتی (بنیان گذار شرکت) به عنوان رییس گروه مشخص شده و مسئول هدایت دانشجویان در طول مدت کارآموزی می‌باشد.

 بعد از همفکری با استاد محترم کارآموزی دکتر سیدین و همچنین مشورت با مسئول کارآموزی شرکت پروژه بازشناسی گفتار فارسی انتخاب شد. من در طول دوره کارآموزی موظف به پیدا کردن بهترین مدل ترانسفرمری برای این امر و پیاده سازی آن بر روی سرور های شرکت بودم. مدل جدید قرار است بجای مدل قدیمی شرکت در نرم افزار نویسا قرار گیرد. در فصل بعدی به مدل انتخاب شده، معماری آن، دادگان استفاده شده و چالش های آن بیان خواهد شد. 
