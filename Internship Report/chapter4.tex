\chapter{جمع‌بندي و نتيجه‌گيري و پیشنهادات}
%%%%%%%%%%%%%%%%%%%%%%%%%%%%%%%%%%%%%%%%%%%
در این فصل با توجه مباحثی که در فصل های قبل بیان شد و تجارب ارزشمندی که کسب شد به نتیجه‌گیری و پیشنهادات این پروژه کارآموزی اشاره خواهد شد.

\section{نتیجه‌گیر و جمع‌بندی}

همانطور که در فصل های گذشته بیان شد مدل های بازشناسی گفتار امروز کاربرد های زیادی دارند اما با اینکه چندین سرویس بازشناسی گفتار فارسی موجود است امّا هیچ کدام از آنها از مدل های جدید خودنگرش\LTRfootnote{Self-attention} استفاده نمی‌کنند؛ این امر باعث شده است در بعضی موارد ضعیف عمل کنند. من در دوره با راهنمایی ارشد پروژه و کمک های استاد راهنمای کارآموزی دکتر سیدین سرپرست کارآموزی دکتر بیراوند موفق و شدم یک مدل بازشناسی گفتار فارسی با دقت بسیار بالا درست کنم.

در نهایت این سرویس بازشناسی گفتار فارسی بر روی سرور های شرکت و سرور های هاگینگ فیس پیاده سازی شد و در حال حاظر نسخه اولیه آن در دسترس عموم می‌باشد. این مدل جدید توانایی درک کلمات جدید و پیچیده را دارد و می‌تواند به درستی رو نویسی کند امکانی که در مدل های قبلی فراهم نبود.

با این حال برای رسیدن به نسخه نهایی و استفاده در صنعت نیازمند استفاده  دیتاست های بزرگتر و با دامنه گفتار گسترده تر می‌باشد. همچنین استفاده از مدل زبانی بزرگ هم میتواند در بهبود عملکرد این مدل نقش به سزایی ایفا کند.


\section{پیشنهادات}
همان طور پیشتر گفته شد تعداد بیشتر دادگان می‌تواند باعث افزایش هرچه بیشتر دقت سرویس شود. به این منظور استفاده از داده های گفتاری که شامل لهجه های مختلف، صدا های پس زمینه مختلف و گوینده هایی با بازه سنی بیشتر است می‌تواند دقت سرویس بازشناسی گفتار فارسی را بسیار افزایش دهد.

پیشنهاد دیگری که می‌توان برای این پروژه ارائه کرد استفاده مدل هایی با معماری بزرگتر و پارامتر های بیشتر می‌باشد. برای این پروژه بخاطر محدودیت در امکانات محسباتی من مجبور شدم که از معماری متوسط ای-برانچفرمر متوسط استفاده کنم که بتوان با وجود حافظه کم واحد پردازش گرافیکی مدل را آموزش دهم. اما اگر سخت افزار های قوی تر با توان محاسباتی و حافظه بیشتر موجود باشد می‌توان مدل ای-برانچفرمر بزرگ را آموزش داد و دقت خروجی را بالا برد.

به عنوان آخرین پیشنهاد می‌توان استفاده مدل های زبانی بزرگ\LTRfootnote{Large Language Model} را مطرح کرد.
تحقیقات جدید نشان می‌دهند که می‌توان یک مدل زبانی بزرگ را بر روی حجم زیادی از دادگان آموزش داد و از آن برای تمام کار های پردازش گفتار و متن استفاده کرد و نتایج خوبی دریافت کرد.\cite{huang2023language} باتوجه به اینکه در بازشناسی گفتار از مدل زبانی و مدل صوت شناسی کنار هم استفاده می‌شود بنظر می‌آید که مدل زبانی قوی می‌تواند ضعف مدل صوت شناسی را جبران کند و خروجی هایی با دقت بالا تولید کند.