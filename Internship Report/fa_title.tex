%% -!TEX root = AUTthesis.tex
% در این فایل، عنوان پایان‌نامه، مشخصات خود، متن تقدیمی‌، ستایش، سپاس‌گزاری و چکیده پایان‌نامه را به فارسی، وارد کنید.
% توجه داشته باشید که جدول حاوی مشخصات پروژه/پایان‌نامه/رساله و همچنین، مشخصات داخل آن، به طور خودکار، درج می‌شود.
%%%%%%%%%%%%%%%%%%%%%%%%%%%%%%%%%%%%
% دانشکده، آموزشکده و یا پژوهشکده  خود را وارد کنید
\faculty{دانشکده مهندسی برق}
% گرایش و گروه آموزشی خود را وارد کنید
\department{گرایش الکترونیک}
% عنوان پایان‌نامه را وارد کنید
\fatitle{محل کارآموزی
\\[.75 cm]
شرکت عصر گویش پرداز}

% نام استاد(ان) راهنما را وارد کنید
\firstsupervisor{دکتر ساناز سیدین}
% \secondsupervisor{استاد راهنمای دوم}
% نام استاد(دان) مشاور را وارد کنید. چنانچه استاد مشاور ندارید، دستور پایین را غیرفعال کنید.
\firstadvisor{دکتر حمزه بیراوند}
\name{پارسا }
% نام خانوادگی نویسنده را وارد کنید
\surname{محمدی}
%%%%%%%%%%%%%%%%%%%%%%%%%%%%%%%%%%
\thesisdate{شهریور 1402}

% چکیده پایان‌نامه را وارد کنید
\fa-abstract
{
مدل‌های بازشناسی خودکار گفتار دسته‌ای از سیستم‌های هوش مصنوعی هستند\LTRfootnote{Automatic Speech Recognition} که برای تبدیل زبان گفتاری به متن نوشتاری طراحی شده‌اند. این مدل‌ها در طیف گسترده‌ای از کاربردها، از جمله خدمات رونویسی، دستیارهای صوتی و دستگاه‌های کنترل‌شده با صدا، نقش مهمی دارند.
\newline
در این پروژه مقاله یکی از جدیدترین و بهترین مدل های بازشناسی گفتار خوانده شده و مدل آن پیاده سازی شده است. این معماری جدید \texttt{E-Branchformer} نام دارد که توانسته است به نتایج بهتری در مقایسه با مدل های \texttt{Conformer} بدست بیاورد. در این پروژه کارآموزی به منظور ایجاد یک مدل بازشناسی گفتار برای زبان فارسی، مدل  \texttt{E-Branchformer} بر روی داده های فارسی مجموعه دادگان کامن ویس\LTRfootnote{Common Voice} به عنوان مدل صوت شناسی\LTRfootnote{Acoustics} آموزش داده شده است و همچنین یک مدل زبانی ترانسفرمری نیز بر روی داده های متنی این پایگاه داده آموزش داده شده است. این مدل نهایی که ترکیب مدل صورت شانسی و زبانی می‌باشد موفق به کسب نرخ خطای کلمات \LTRfootnote{WER: Word Error Rate} 3 درصد بر روی داده های تست کامن ویس شده است که با توجه به حجم کم دادگان نتیجه بسیار خوبی می‌باشد. در مرحله بعد این مدل بر روی سرور های هاگینیگ فیس\LTRfootnote{Huggingface} پیاده سازی شده است و مدل به صورت برخط \LTRfootnote{Online} برای عموم قابل دسترس می‌باشد.}


% کلمات کلیدی پایان‌نامه را وارد کنید
\keywords{پردازش گفتار , بازشناسی گفتار فارسی, پردازش زبان طبیعی}



\AUTtitle
%%%%%%%%%%%%%%%%%%%%%%%%%%%%%%%%%%
\vspace*{7cm}
\thispagestyle{empty}
\begin{center}
\includegraphics[height=5cm,width=12cm]{Images/besm.jpg}
\end{center}